\documentclass{ees}

\shorttitle{Sum offensa}

\begin{document}

\eesTitlePage

\eesCriticalReport{
  – & –    & org & All bass figures habe been added by the editor. \\
  1 & –    & –   & \textit{Fine} in bar 158 and \textit{Segno} before bar 33 have been added by the editor. \\
    & –    & vl 1 & Grace notes have been added by the editor in bars 84 (4th \quarterNote), 85 (3rd \quarterNote), 87 (both), 88 (4th \quarterNote), 134 (4th \quarterNote), 136 (3rd \quarterNote), 138 (both), 139 (1st \quarterNote), 140 (3rd \quarterNote), 141 (both), 144, 156 (2nd \quarterNote). \\
    & –    & vl 2 & Grace notes have been added by the editor in bars 84, 87, 88, 134, 136 (3rd \quarterNote), 139, 140 (3rd \quarterNote), 144. \\
    & –    & S    & Grace notes have been added by the editor in bars 86 (2nd \quarterNote), 87 (1st \quarterNote), 89 (3rd \quarterNote). \\
    & 9    & vla, org & last \eighthNote\ in \B1: e′8 \\
    & 10–13 & vl 1 & bars in \B1: b″1–a″1–g″1 \\
    & 23   & vl & 11th and 15th \sixteenthNote\ in \B1: a″16 and g″16, respectively \\
    & 39   & vla, org  & 2nd \halfNote\ in \B1: e2 \\
    & 63–72 & vl & The same rhythm as in bar 62 might have been intended. \\
    & 73 & 4th \quarterNote\ in \B1: \sharp f″4 \\
    & 89   & S    & rhythm of 1st and 3rd \quarterNote\ in \B1: \sixteenthNote–\sixteenthNote–\eighthNote \\
    & 105 & vla, org & 2nd \quarterNote\ in \B1: \crotchetRest \\
    & 108 & vl 2 & 1st \halfNote\ missing in \B1 \\
    & 143  & vla, org & bar in \B1: a1 \\
    & 172  & vla, org & 1st \halfNote\ in \B1: B2 \\
    & 175  & vla, org & 2nd \quarterNote\ in \B1: d4 \\
  \midrule
  2 & 27 & org & 3rd \quarterNote\ in \B1: a8–a8 \\
  \midrule
  3 & 28 & vl 2 & 1st \eighthNote\ in \B1: a′16–g′16 \\
    & 29 & vl 1, S & last \eighthNote\ in \B1: b′16–a′16 \\
    & 29 & vl 2 & 2nd \quarterNote\ in \B1: a′32–e′16.–e′32–d′16. \\
    & 40 & S & rhythm of 2nd \eighthNote\ in \B1: 2 × \sixteenthNote \\
    & 42 & S & rhythm of 2nd to 4th \eighthNote\ in \B1: 6 × \sixteenthNote \\
    & 60 & vl 1 & 2nd \eighthNote\ in \B1: a″16–\sharp f″16–d″16 \\
    & 80 & vl 2 & 1st \eighthNote\ in \B1: d′8 \\
    & 99 & vl 2 & 3rd \eighthNote\ in \B1: g′8 \\
    & 115–117 6 org & bars in \B1: 3 × e2 \\
    & 122–124 & org bars in \B1: e2, a4–g4, and \sharp f2 \\
    & 126 & org & bar in \B1: e4–g4 \\
  \midrule
  4 25–27 & vla & in \B1 unison with org \\
    & 73f & vl & rhythm of 2nd \eighthNote\ in \B1: \sixteenthNote–\sixteenthNote \\
    & 104f & vl 1 & 2nd \eighth in \B1: a″16–g″16 \\
    & 104f & vl 2 & 2nd \eighth in \B1: \sharp f″16–d″16 \\
}

\eesToc{}

\eesScore

\end{document}


Sum offensa, sum irata,
eia fortis amor meus,
arcum tende, vibra telum,
cadat impius, cadat reus,
timor anime fatalis.
Cordis noto arrideat caelum,
me furore ac odio armata,
percit illitam crudelis,
sic exinde ope firmata,
fides erit immortalis.


Quae loquor quae deliro
timor non est qui amare caeli turbat in me.
Heu dum aspiro ad summum Dei favore,
miserum cor non vides quam brevis,
quam infirma sit in te fides?
Quid nunc agendum? Dic!
Ah! Respondes?
Cum vera fervet amor,
semper timore rigat,
sed nunc maior ab ipsa affectus viget.
Ita sit, ergo spera exora plange clama,
fidem confirma tuam time ed ama.

Dum Philomela in ramo
cantando dicit amo,
per auras dulce penas
metus infesti narrat
mesta gemendo in se.
Sic quando umbra timoris
fit causa mei doloris,
voces ad caelum spargit
metu et amore plenas,
afflictum cor in me,
afflictum cor in te.

Alleluia.
